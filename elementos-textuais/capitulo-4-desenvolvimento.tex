\chapter{Desenvolvimento da aplicação AutoForm}
% ---
Neste capitulo é apresentado o desenvolvimento do projeto, onde são apresentados, a arquitetura do sistema, as tecnologias utilizadas,e a implementação do sistema.

\subsection{Processo de produção}
O projeto teve sua fase inicial, após o 1º encontro com o cliente Tiago Costa e o coorientador Márcio Lemos no \textit{campus} Osório do IFRS, conforme demonstrado a necessidade pela BMRS de uma aplicação que pudesse ser utilizada para auxiliar no processo de registro do AEL no SIGMA, a fim de agilizar o processo de preenchimento e automatizar a geração do arquivo, bem como resolver outras dificuldades encontradas pelos operadores.
Após a definição de contexto do AEL por parte da BM RS, foi realizado um estudo sobre o assunto, para que fosse possível entender o problema e propor uma solução viável e eficaz. 

Com a elicitação dos requisitos concluída, ocorreu a prototipação das telas da aplicação, utilizando a ferramenta de \textit{design de interface } de usuário \cite{figma}, a fim de validar se o visual e o fluxo da aplicação iriam atender o proposito da instituição.

Após a validação do protótipo, foi realizada a escolha das tecnologias que seriam utilizadas no desenvolvimento da aplicação, sendo elas o \textit{framework} \cite{React22:online}, para o desenvolvimento \textit{frontend}, e o \textit{framework} NodeJS \cite{Nodejs} para o desenvolvimento do \textit{backend} da aplicação web, juntamente com o banco de dados não relacional \cite{MongoDBA45:online}, para o armazenamento dos dados.

Com as tecnologias definidas, foi realizado o desenvolvimento da aplicação web e hospedado em um servidor, para que fosse possível realizar os testes e validações com o cliente, e assim, realizar as correções necessárias. 

Portanto após a primeira versão da aplicação estar disponível na \textit{internet} as adaptações necessárias e atualizações eram organizadas e relatadas mediante troca de mensagens com o cliente.

Para permitir que fosse possível realizar atualizações em produção na aplicação foi utilizada a ferramente\cite{git}, juntamente com a plataforma de hospedagem de código fonte \cite{github}, para que fosse possível realizar o controle de versões e o versionamento do código fonte da aplicação, garantindo sempre uma versão estável e disponivel em produção, enquanto era viável desenvolver paralelamente novas funcionalidades e valida-las com o cliente.


\subsection{Aplicação Web }


% Este capítulo possui também exemplos de como inserir código no texto.
% % ---
% \section{Vestibulum ante ipsum primis in faucibus orci luctus et ultrices
% posuere cubilia Curae}
% % ---

% \lipsum[21-22]

% \section{Inserindo código no texto}

% Abaixo são apresentados exemplos de códigos inseridos no texto usando o pacote listings.

% %Inserindo código diretamente no texto
% \begin{lstlisting}[language=c, caption={Exemplo de código C}, upquote=true]
% #include <stdio.h>

% int main(int argc, const char * argv[]) {
% struct pessoa{
% char nome[20];
% char sobreNome[20];
% unsigned short idade;
% char cpf[15];
% }p1;
% //struct pessoa p1;

% printf("Digite o nome: ");
% scanf("%s",p1.nome);
% printf("Digite o sobrenome: ");
% scanf("%s",p1.sobreNome);
% printf("Digite a idade: ");
% scanf("%hu",&p1.idade);
% printf("Digite o CPF: ");
% scanf("%s",p1.cpf);

% printf("\nO nome da pessoa eh: %s", p1.nome);
% printf("\nO sobrenome da pessoa eh: %s", p1.sobreNome);
% printf("\nA idade da pessoa eh: %d", p1.idade);
% printf("\nO CPF da pessoa eh: %s\n", p1.cpf);
% }
% \end{lstlisting}

% \lipsum[1]

% %Ao invés de inserir código diretamente no texto, é recomendável importar um arquivo com o código

% %Importanto arquivo de código
% \lstinputlisting[language=Java,caption={Exemplo de código java}]{codigos/Minhaclasse.java}

% O método abaixo mostra a string mensagem recebida como parametro e retorna um inteiro digitado pelo usuario. Através de tratamento de exceções, o método executará até que o usuário digite um inteiro válido.

% %Para inserir apenas algumas linhas do arquivo
% \lstinputlisting[language=Java, caption={Método LerInteiro}, firstline=39, lastline=54]{codigos/Ferramentas.java}

% \lipsum[2-3]