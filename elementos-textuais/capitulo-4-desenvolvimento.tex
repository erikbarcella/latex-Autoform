\chapter{Desenvolvimento}
% ---
Neste capitulo você deve explicar o que fez, seu sistema/protótipo/etc... Título livre.\newline 

Este capítulo possui também exemplos de como inserir código no texto.
% ---
\section{Vestibulum ante ipsum primis in faucibus orci luctus et ultrices
posuere cubilia Curae}
% ---

\lipsum[21-22]

\section{Inserindo código no texto}

Abaixo são apresentados exemplos de códigos inseridos no texto usando o pacote listings.

%Inserindo código diretamente no texto
\begin{lstlisting}[language=c, caption={Exemplo de código C}, upquote=true]
#include <stdio.h>

int main(int argc, const char * argv[]) {
struct pessoa{
char nome[20];
char sobreNome[20];
unsigned short idade;
char cpf[15];
}p1;
//struct pessoa p1;

printf("Digite o nome: ");
scanf("%s",p1.nome);
printf("Digite o sobrenome: ");
scanf("%s",p1.sobreNome);
printf("Digite a idade: ");
scanf("%hu",&p1.idade);
printf("Digite o CPF: ");
scanf("%s",p1.cpf);

printf("\nO nome da pessoa eh: %s", p1.nome);
printf("\nO sobrenome da pessoa eh: %s", p1.sobreNome);
printf("\nA idade da pessoa eh: %d", p1.idade);
printf("\nO CPF da pessoa eh: %s\n", p1.cpf);
}
\end{lstlisting}

\lipsum[1]

%Ao invés de inserir código diretamente no texto, é recomendável importar um arquivo com o código

%Importanto arquivo de código
\lstinputlisting[language=Java,caption={Exemplo de código java}]{codigos/Minhaclasse.java}

O método abaixo mostra a string mensagem recebida como parametro e retorna um inteiro digitado pelo usuario. Através de tratamento de exceções, o método executará até que o usuário digite um inteiro válido.

%Para inserir apenas algumas linhas do arquivo
\lstinputlisting[language=Java, caption={Método LerInteiro}, firstline=39, lastline=54]{codigos/Ferramentas.java}

\lipsum[2-3]