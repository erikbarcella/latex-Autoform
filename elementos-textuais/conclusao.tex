\chapter{Conclusão}
O objetivo principal deste trabalho foi desenvolver uma solução que auxiliasse de forma eficaz os operadores da Brigada Militar ao realizar o processo de registro de produtos controlados no SIGMA através do AEL.

Para isso, foram levantadas as informações necessárias, através da elicitação de requisitos , pesquisa exploratória e analise documental para orientar á pesquisa em busca de uma solução eficiente que fosse segura,tivesse disponibilidade de acesso fácil, não armazenasse dados sensíveis, e ainda fosse capaz de minimizar o esforço empregado pelo operador para gerar o arquivo eletrônico.

Portanto o AutoForm, atendeu aos objetivos específicos estabelecidos proporcionando um preenchimento facilitado, através de uma \textit{interface} intuitiva e amigável,disponível de fácil acesso através da \textit{web} e de forma segura, permitindo que diversos usuários realizem acesso simultâneo e ainda, não armazenando dados sensíveis, pois o arquivo é gerado no computador do usuário.

A aplicação possibilita o cadastro de armas para todos os seus utilizadores, bem como o gerenciamento de usuários para o administrador. Viabilizando a supervisão por parte do administrador sobre quais pessoas têm acesso à aplicação.

\subsection{Resultados finais}

Sendo assim capaz de reduzir o tempo de preenchimento do formulário em 38.46\% ao realizar a seleção de uma arma cadastrada para autopreencher 15 campos e inserir informações em todos os 24 restantes incluindo opcionais, essa porcentagem ainda é maior se preencher somente os demais 17 campos obrigatórios ficando então 46.87\%, e 100\% respectivamente para a geração do arquivo eletrônico que antes era realizado manualmente.

Logo pode-se concluir que através deste trabalho foi alavancado ainda mais a excelência operacional da Brigada Militar, contribuindo para a regulamentação dos produtos controlados, e consequentemente para a segurança pública do estado do Rio Grande do Sul.

\section{Trabalhos futuros}
Nesta seção são descritos alguns trabalhos futuros que podem ser realizados para melhorar a aplicação.

Entre estas atualizações fica a implementação de alguns campos fixos no AEL como: tipo proprietário , orgão, orgão que publicou e profissão do proprietário, que não foram implementados por não serem obrigatórios, mas que podem ser definidos futuramente por se tratar de informações fixas inerentes a instituição, logo irá minimizar ainda mais a quantidade de campos a serem preenchidos.

Na página de configurações adicionar uma opção para o administrador poder selecionar algum outro usuário comum e inserir essa permissão especial, para que assim possa ser delegado a outra pessoa a responsabilidade de gerenciar os demais usuários. Também a possibilidade de um botão com a opção para cadastrar novos usuários através desta página.

Na página do AEL, adicionar um model que permita visualizar e editar os registros temporários já salvos, para que assim o usuário possa alterar algum dado que tenha sido preenchido incorretamente, ou até mesmo excluir o registro temporário.


% ---

