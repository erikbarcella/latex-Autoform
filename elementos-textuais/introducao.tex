% ----------------------------------------------------------
% Introdução (com numeração)
% ----------------------------------------------------------
\chapter{Introdução}
% ----------------------------------------------------------
O Sistema de Gerenciamento Militar de Armas (SIGMA) é um sistema informatizado utilizado
pelo Comando do Exército para gerenciar o registro e transferência de armas de fogo no
Brasil. 
A Brigada Militar do Rio Grande do Sul é reconhecida pelo seu alto nível de profissionalismo e 
adesão às boas práticas disciplinares.
Complementando essas características, a utilização de uma ferramenta especializada tem o potencial de 
aprimorar ainda mais os processos administrativos da corporação.
Ao adotar essa ferramenta, espera-se obter maior eficiência, confiabilidade e alinhamento com as 
exigências do Exército Brasileiro. Embora ainda não tenhamos detalhado a ferramenta específica neste projeto,
é importante destacar que sua implementação visa otimizar os procedimentos administrativos da Brigada Militar, 
proporcionando resultados mais eficazes e alavancando a excelência operacional da instituição.
Atualmente, na Brigada Militar do Rio Grande do Sul, o setor interno é responsável por controlar e 
manter diversos processos administrativos, inclusive interligados a outros órgãos públicos. 
Um dos desafios enfrentados pelos operadores que realizam esta atividade é a digitação manual de dados em um 
arquivo de texto, chamado DataList, seguido pela formatação e adequação do documento ao modelo padrão 
estabelecido pelo Exército Brasileiro, que possui regras específicas de indexação das informações
Para aprimorar a eficiência -nesse processo.
\index{citações!diretas}Utilize o ambiente \texttt{citacao} para incluir
citações diretas com mais de três linhas:
\begin{citacao}
motivação em simplificar e melhorar os fluxos de trabalho, levar a reduzir custos operacionais e implementar novas aplicações mais rápidas através da automação, programação e gestão das transferências de arquivo\cite[5.3]{AndradeJunior}
\end{citacao}


\section{Justificativa}

\section{Objetivos}

\subsection{Objetivos Gerais}

\subsection{Objetivos específicos}
