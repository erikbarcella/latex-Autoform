% ----------------------------------------------------------
% Introdução (com numeração)
% ----------------------------------------------------------
\chapter{Introdução}
% ----------------------------------------------------------

O Sistema de Gerenciamento Militar de Armas (SIGMA) é um sistema informatizado utilizado como ferramenta de controle e rastreamento para gerenciar o registro e transferência de armas de fogo, munições e demais produtos controlados de competência do Comando do Exército em todo o território Brasileiro 
\cite{ExércitoBrasileiro}.

Sua criação e implementação foram conduzidas pelo Ministério da Defesa, em coordenação com o Comando do Exército. O SIGMA tem a finalidade de administrar os registros de armas de propriedade particular pertencentes a diversos grupos, incluindo as armas de fogo de integrantes das Forças Armadas, das Polícias Militares e dos órgãos de segurança pública, além de empresas de segurança privada e colecionadores de armas. Em essência, o SIGMA atua como um sistema centralizado de gerenciamento de informações sobre armas de fogo sob posse dessas entidades e indivíduos, contribuindo para a regulamentação e controle no contexto nacional
\cite{ExércitoBrasileiro}.


Na Brigada Militar do Rio Grande do Sul, o setor administrativo de cadastros e registros é responsável por controlar e manter diversos processos que abrangem uma variedade de assuntos relacionados às atividades e operações da instituição militar, inclusive interligados a outros órgãos públicos
\cite{bmDepartamentoAdministrativo}.

Portanto um dos processos administrativos mantidos pela corporação é a geração do AEL, onde atualmente esta tarefa é realizada pelos seus operadores de forma manual preenchendo os dados necessários em um arquivo de texto, seguido da formatação e adequação do documento ao modelo padrão estabelecido pelo Exército Brasileiro, que possui regras específicas de indexação das informações conforme especificado no manual de preenchimento do arquivo eletrônico (anexo \ref{sec:anexoA1}). 

Após a geração do arquivo eletrônico, o mesmo é submetido para avaliação da Diretoria de Fiscalização de Produtos Controlados do Exército, que irá verificar as informações contidas no arquivo e fazer o \textit{upload} para o SIGMA, após validação do sistema, em caso de sucesso do cadastro, é retornado um número de identificação único gerado pelo SIGMA, o qual é utilizado pelo exército para identificar o respectivo registro e todas suas informações dentro do SIGMA quando necessário \cite{ExércitoBrasileiro}.

E, utilizado pela Brigada Militar o número de identificação gerado pelo SIGMA como retorno de que o cadastro daquele processo foi efetivado na base de dados e finalizado, conforme evidenciado seção 3.4 alínea C do anexo \ref{sec:anexoA4}.

Entretanto, no momento atual o AEL é formatado manualmente pelo operador após o preenchimento, implicando em uma maior complexidade na execução dessa tarefa. A proposta deste trabalho é desenvolver uma aplicação web que automatize o processo de geração do arquivo eletrônico visando simplificar significativamente o procedimento, através do preenchimento facilitado proporcionando resultados mais eficazes e alavancando a excelência operacional da instituição.


\section{Justificativa}

A divisão interna da BM RS desempenha o papel crucial de supervisionar e gerir uma série de procedimentos administrativos que abrangem uma ampla gama de assuntos relacionados às atividades e operações da instituição militar \cite{bmDepartamentoAdministrativo}.

Uma das atividades administrativas sob a responsabilidade da corporação é registrar e manter atualizado o cadastro, transferência de armas de fogo e demais produtos regulados sob a jurisdição do Comando do Exército no estado do Rio Grande do Sul, através do envio de documento eletrônico para a DFPC destinado a registrar no SIGMA \cite{ExércitoBrasileiro}.

Neste momento, a execução dessa atividade é realizada de forma manual por operadores da BM RS, os quais inserem os dados essenciais em um documento de texto. Além, de precederem com a formatação do documento de acordo com as diretrizes estipuladas pelo Exército Brasileiro, que estabelece regras específicas para a indexação das informações, conforme minuciosamente delineado no manual de preenchimento de arquivo eletrônico (anexo \ref{sec:anexoA4}). 

Nesse sentido, foi estipulado o desenvolvimento uma aplicação digital a fim de simplificar notavelmente o processo de preenchimento e geração do AEL na Brigada Militar, de forma que esta plataforma contribua de maneira eficiente e decisiva.
% \begin{citacao}
%     Motivação em simplificar e melhorar os fluxos de trabalho, levar a reduzir custos operacionais e implementar novas aplicações mais rápidas através da automação, programação e gestão das transferências de arquivo\cite{AndradeJunior}
%     \end{citacao}

\section{Objetivos}
Os objetivos esperados são apresentados nesta seção abaixo, divididos em geral e específicos. 

\subsection{Objetivo Geral}
O objetivo deste trabalho é desenvolver uma aplicação web que contribua de forma eficaz para o preenchimento e geração do AEL, buscando otimizar o fluxo do processo para a Brigada Militar, através de preenchimento facilitado e geração automática do arquivo conforme o padrão pré estabelecido pelo anexo D da portaria 136 de 8 de novembro de 2019 do Exército Brasileiro (anexo \ref{sec:anexoA4}).

\subsection{Objetivos específicos}
\begin{itemize}
    \item Realizar o levantamento de requisitos necessários para o desenvolvimento da aplicação web. 
    \item Desenvolver o cadastro de armas, persistindo no banco de dados.
    \item Autopreencher os campos do formulário com informações da arma quando selecionada uma previamente cadastrada pelo operador.
    \item Permitir o gerenciamento de usuários através de um acesso especial com permissão de administrador do sistema.
    \item Facilitar o  fluxo para o operador no preenchimento do formulário web, retornando para o usuário possíveis erros e regras específicas que devem ser seguidas.
    \item Automatizar a geração do arquivo.
\end{itemize}  
