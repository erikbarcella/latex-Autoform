% ----------------------------------------------------------
% Introdução (com numeração)
% ----------------------------------------------------------
\chapter{Introdução}
% ----------------------------------------------------------

O Sistema de Gerenciamento Militar de Armas (SIGMA) é um sistema informatizado utilizado como ferramenta de controle e rastreamento para gerenciar o registro e transferência de armas de fogo, munições e demais produtos controlados de competência do Comando do Exército em todo o território Brasileiro 
\cite{ExércitoBrasileiro}.

Sua criação e implementação foram conduzidas pelo Ministério da Defesa, em coordenação com o Comando do Exército. O SIGMA tem a finalidade de administrar os registros de armas de propriedade particular pertencentes a diversos grupos, incluindo as armas de fogo de integrantes das Forças Armadas, das Polícias Militares e dos órgãos de segurança pública, além de empresas de segurança privada e colecionadores de armas. Em essência, o SIGMA atua como um sistema centralizado de gerenciamento de informações sobre armas de fogo sob posse dessas entidades e indivíduos, contribuindo para a regulamentação e controle no contexto nacional.
\cite{ExércitoBrasileiro}


Na Brigada Militar do Rio Grande do Sul, o setor interno é responsável por controlar e 
manter diversos processos administrativos que abrangem uma variedade de assuntos relacionados às atividades e operações da instituição militar, inclusive interligados a outros órgãos públicos. 
\cite{bmDepartamentoAdministrativo}

Portanto um dos processos administrativos mantidos pela corporação é a geração do AEL, onde atualmente esta tarefa é realizada pelos seus operadores de forma manual preenchendo os dados necessários em um arquivo de texto, seguido da formatação e adequação do documento ao modelo padrão estabelecido pelo Exército Brasileiro, que possui regras específicas de indexação das informações conforme especificado no manual de preenchimento do arquivo eletrônico conforme (anexo \ref{sec:anexoA1})   

Após a geração do arquivo eletrônico , o mesmo é submetido para avaliação da Diretoria de Fiscalização de Produtos Controlados do Exército e caso validado é registrado no SIGMA\cite{ExércitoBrasileiro}

Portanto a proposta deste trabalho é desenvolver uma aplicação web que automatize o processo de geração do AEL, onde o operador irá preencher os dados necessários em um formulário web, e após a submissão do mesmo, o sistema irá gerar o arquivo eletrônico no formato especificado pelo Exército Brasileiro, e por fim o operador irá submeter o arquivo eletrônico para avaliação da Diretoria de Fiscalização de Produtos Controlados do Exército.

Ao adotar essa ferramenta, espera-se obter maior eficiência, confiabilidade e alinhamento com as 
exigências do Exército Brasileiro. Embora ainda não tenhamos detalhado a ferramenta específica neste projeto,
é importante destacar que sua implementação visa otimizar os procedimentos administrativos da Brigada Militar, 
proporcionando resultados mais eficazes e alavancando a excelência operacional da instituição.
Atualmente, na Brigada Militar do Rio Grande do Sul, o setor interno é responsável por controlar e 
manter diversos processos administrativos, inclusive interligados a outros órgãos públicos. 
Um dos desafios enfrentados pelos operadores que realizam esta atividade é a digitação manual de dados em um 
arquivo de texto, chamado DataList, seguido pela formatação e adequação do documento ao modelo padrão 
estabelecido pelo Exército Brasileiro, que possui regras específicas de indexação das informações
Para aprimorar a eficiência -nesse processo.
\index{citações!diretas}Utilize o ambiente \texttt{citacao} para incluir
citações diretas com mais de três linhas:
\begin{citacao}
motivação em simplificar e melhorar os fluxos de trabalho, levar a reduzir custos operacionais e implementar novas aplicações mais rápidas através da automação, programação e gestão das transferências de arquivo\cite[5.3]{AndradeJunior}
\end{citacao}

\section{Justificativa}

\section{Objetivos}

\subsection{Objetivos Gerais}

\subsection{Objetivos específicos}
