\chapter{Trabalhos Relacionados}
% ---
% ---
\section{TAF- teste de aptidão física da brigada militar do rio grande do sul}
Um estudo feito por gabriela machado durante o curso de Analise e desenvolvimento de sistemas no IFRS- campus Osório em 2018, o taf é uma avaliação física que visa avaliar a aptidão física dos candidatos a ingressar na brigada militar do rio grande do sul. É uma etapa importante do processo seletivo, e tem como objetivo verificar se os candidatos possuem as condições físicas mínimas exigidas para desempenhar as atividades do cargo. Se concentrando em diferentes aspectos, como a validade e a confiabilidade do teste, a relação entre os resultados do taf e o desempenho dos candidatos nas atividades militares, os fatores que influenciam o desempenho dos candidatos no taf, entre outros.

\section{Melhoria de processo pelo BPM, aplicação no setor publico}
O artigo de claudio josé muller e isadora cidade mariano apresenta um relato de uma aplicação da metodologia bpm (business process management), que foi realizada em quatro etapas: (i) planejamento das atividades do bpm; (ii) mapeamento do processo escolhido; (iii) proposta de melhorias e comparação entre o processo atual e o proposto. A metodologia foi adaptada para o contexto de uma organização pública e esta abordagem foi utilizada para modernizar o processo de controle de trânsito animal no brasil. A partir da análise do processo atual foram propostas melhorias a fim de otimizar recursos, melhorar a confiabilidade e aumentar a satisfação de clientes.

\section{Automação de processos manuais}
Por Luiz Roberto de Andrade Júnior as instituições financeiras possuem uma longa história de solução de problemas na área de informática através da criação de novas ferramentas e tecnologias de automação de processos de trabalho, para garantir uma entrega mais rápida de suas tarefas. O presente trabalho apresenta a automação de processos hoje executados manualmente e com a ajuda da ferramenta CONTROL-M desenvolvida pela empresa BMC software. Foi adquirida uma ferramenta capaz de verificar o estado de execução de tarefas agendadas, com a análise de seus resultados gerados. A verificação é realizada por meio de critérios de validação customizáveis pelo usuário da ferramenta em questão
% ---
