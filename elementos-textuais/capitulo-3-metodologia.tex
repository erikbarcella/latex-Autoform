\chapter{Metodologia}
% ---
% ---
Informações sobre a pesquisa aplicada...
A metodologia utilizada para a pesquisa bibliográfica no presente estudo envolveu a busca e análise de fontes diversas, como sites, artigos científicos e livros, disponibilizados em formato digital na web. Essa abordagem permitiu a coleta de informações relevantes e atuais para a construção do referencial teórico do estudo.
Através da revisão da literatura disponível, é possível identificar conceitos, teorias, abordagens e práticas relacionadas ao assunto estudado.
Os sites consultados durante a pesquisa bibliográfica podem incluir sites de instituições acadêmicas, bases de dados científicas, periódicos online, portais de pesquisa e outras fontes confiáveis na web. Essas fontes são importantes para acessar artigos científicos, relatórios, teses, dissertações e outras publicações acadêmicas.


\section{Levantamento de Requisitos}

\subsection{Requisitos Funcionais}

\subsection{Requisitos Não Funcionais}

% % ---
% \section{Tabelas}
% % ---

% \index{tabelas}A \autoref{tab-nivinv} é um exemplo de tabela construída em
% \LaTeX.

% \begin{table}[htb]
% \ABNTEXfontereduzida
% \caption[Níveis de investigação]{Níveis de investigação.}
% \label{tab-nivinv}
% \begin{tabular}{p{2.6cm}|p{6.0cm}|p{2.25cm}|p{3.40cm}}
%   %\hline
%    \textbf{Nível de Investigação} & \textbf{Insumos}  & \textbf{Sistemas de Investigação}  & \textbf{Produtos}  \\
%     \hline
%     Meta-nível & Filosofia\index{filosofia} da Ciência  & Epistemologia &
%     Paradigma  \\
%     \hline
%     Nível do objeto & Paradigmas do metanível e evidências do nível inferior &
%     Ciência  & Teorias e modelos \\
%     \hline
%     Nível inferior & Modelos e métodos do nível do objeto e problemas do nível inferior & Prática & Solução de problemas  \\
%    % \hline
% \end{tabular}
% \legend{Fonte: \citeonline{van86}}
% \end{table}

% Já a \autoref{tabela-ibge} apresenta uma tabela criada conforme o padrão do
% \citeonline{ibge1993} requerido pelas normas da ABNT para documentos técnicos e
% acadêmicos.

% \begin{table}[htb]
% \IBGEtab{%
%   \caption{Um Exemplo de tabela alinhada que pode ser longa
%   ou curta, conforme padrão IBGE.}%
%   \label{tabela-ibge}
% }{%
%   \begin{tabular}{ccc}
%   \toprule
%    Nome & Nascimento & Documento \\
%   \midrule \midrule
%    Maria da Silva & 11/11/1111 & 111.111.111-11 \\
%   \midrule 
%    João Souza & 11/11/2111 & 211.111.111-11 \\
%   \midrule 
%    Laura Vicuña & 05/04/1891 & 3111.111.111-11 \\
%   \bottomrule
% \end{tabular}%
% }{%
%   \fonte{Produzido pelos autores.}%
%   \nota{Esta é uma nota, que diz que os dados são baseados na
%   regressão linear.}%
%   \nota[Anotações]{Uma anotação adicional, que pode ser seguida de várias
%   outras.}%
%   }
% \end{table}
% ---


