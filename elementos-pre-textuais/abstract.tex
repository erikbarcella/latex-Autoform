\begin{resumo}[Abstract]
 \begin{otherlanguage*}{english}
  The Rio Grande do Sul Military Brigade is responsible for generating an electronic document called AEL that includes data on weapons registered in the state, and forwarding it to the Army's Controlled Products Inspection Directorate for registration in the Military Weapons Management System (SIGMA) . Due to the demand presented
by BM RS for a system that supports the execution of this process, the development of this web application, called AutoForm, developed with the JavaScript language was suggested
in conjunction with the React and NodeJS frameworks. Using strategies and methodologies
that will be addressed during this research, to facilitate the completion of information by the
operator, optimize the execution time of this task, increase efficiency, and take into account all
necessary requirements for generating the AEL, ensuring that it is complete and correct before
to be submitted to SIGMA

   \vspace{\onelineskip}
 
   \noindent 
   \textbf{Keywords}: Electronic Batch File, SIGMA, Military Brigade, Web Application, React, NodeJS, JavaScript%alterar para as palavras-chave do trabalho
 \end{otherlanguage*}
\end{resumo}