%% Customizacoes do abnTeX2 (http://abnTeX2.googlecode.com) para o IFRS Campus Osorio v1.1
%% Por Bruno Fernandes (bruno.fernandes@osorio.ifrs.edu.br)
%% O modelo mais atualizado está disponível em bit.ly/ADSLaTeX
%%
%% abtex2-modelo-trabalho-academico.tex, v-1.9.6 laurocesar
%% Copyright 2012-2016 by abnTeX2 group at http://www.abntex.net.br/ 
%%
%% This work may be distributed and/or modified under the
%% conditions of the LaTeX Project Public License, either version 1.3
%% of this license or (at your option) any later version.
%% The latest version of this license is in
%%   http://www.latex-project.org/lppl.txt
%% and version 1.3 or later is part of all distributions of LaTeX
%% version 2005/12/01 or later.
%%
%% This work has the LPPL maintenance status `maintained'.
%% 
%% The Current Maintainer of this work is the abnTeX2 team, led
%% by Lauro César Araujo. Further information are available on 
%% http://www.abntex.net.br/
%%
%% This work consists of the files abntex2-modelo-trabalho-academico.tex,
%% abntex2-modelo-include-comandos and abntex2-modelo-references.bib
%%

% ------------------------------------------------------------------------
% ------------------------------------------------------------------------
% abnTeX2: Modelo de Trabalho Academico (tese de doutorado, dissertacao de
% mestrado e trabalhos monos em geral) em conformidade com 
% ABNT NBR 14724:2011: Informacao e documentacao - Trabalhos academicos -
% Apresentacao
% ------------------------------------------------------------------------
% ------------------------------------------------------------------------

\documentclass[
	% -- opções da classe memoir --
	12pt,				% tamanho da fonte
	openright,			% capítulos começam em pág ímpar (insere página vazia caso preciso)
	% twoside,			% para impressão em recto e verso. Oposto a oneside. FRENTE E VERSO
    oneside,			% para impressão em apenas um lado. APENAS FRENTE
	a4paper,			% tamanho do papel. 
	% -- opções da classe abntex2 --
	chapter=TITLE,		% títulos de capítulos convertidos em letras maiúsculas
	section=TITLE,		% títulos de seções convertidos em letras maiúsculas
	% -- opções do pacote babel --
	english,			% idioma adicional para hifenização
	french,				% idioma adicional para hifenização
	spanish,			% idioma adicional para hifenização
	brazil				% o último idioma é o principal do documento
	]{abntex2}


\usepackage{customizacoes-ifrs-osorio}
\usepackage[expansion=false]{microtype}
\usepackage{float}
% ---
% Pacotes básicos 
% ---
\usepackage{longtable}
\usepackage{wasysym}
\usepackage[table]{xcolor}
\usepackage{tgtermes}		
\usepackage[T1]{fontenc}		% Selecao de codigos de fonte.
\usepackage[utf8]{inputenc}		% Codificacao do documento (conversão automática dos acentos)
\usepackage{lastpage}			% Usado pela Ficha catalográfica
\usepackage{indentfirst}		% Indenta o primeiro parágrafo de cada seção.
\usepackage{color}				% Controle das cores
\usepackage{graphicx}			% Inclusão de gráficos
\usepackage{microtype} 			% para melhorias de justificação
\renewcommand{\ABNTEXchapterfont}{\fontfamily{ptm}\fontseries{sbc}\selectfont}
% ---

% ----------------------------------------------
% Configuração das fontes
% ----------------------------------------------
% Algumas configurações de fontes para capitulos e seções
\renewcommand{\ABNTEXchapterfontsize}{\normalsize\bfseries}
\renewcommand{\ABNTEXpartfontsize}{\ABNTEXchapterfontsize}
\renewcommand{\ABNTEXsectionfontsize}{\normalsize}
\renewcommand{\ABNTEXsubsectionfontsize}{\normalsize\bfseries}
\renewcommand{\ABNTEXsubsubsectionfont}{\slshape\bfseries}
\renewcommand{\ABNTEXsubsubsubsectionfont}{\slshape}

% ---
% Pacotes adicionais, usados apenas no âmbito do Modelo Canônico do abnteX2
% ---
\usepackage{lipsum}				% para geração de dummy text
% ---

%pacote para inserir blocos de código
\usepackage{listings}
\usepackage{color}

\usepackage{textcomp}

\usepackage[font=small]{caption}

% ---
% Pacotes de citações
% ---
\usepackage[brazilian,hyperpageref]{backref}	 % Paginas com as citações na bibl
\usepackage[alf]{abntex2cite}	% Citações padrão ABNT


% --- 
% CONFIGURAÇÕES DE PACOTES
% --- 

%Configurações do pacote listings
%New colors defined below
\definecolor{codegreen}{rgb}{0,0.6,0}
\definecolor{codegray}{rgb}{0.5,0.5,0.5}
\definecolor{codered}{rgb}{0.8,0.1,0.3}
\definecolor{backcolour}{rgb}{0.96,0.96,0.93}

%Code listing style named "mystyle"
\lstdefinestyle{mystyle}{
	backgroundcolor=\color{backcolour},   
	commentstyle=\color{codegreen},
	keywordstyle=\bfseries\color{blue},
	numberstyle=\tiny\color{codegray},
	stringstyle=\color{codered},
	basicstyle=\footnotesize\ttfamily,
	breakatwhitespace=false,         
	breaklines=true,                 
	captionpos=t, 
	keepspaces=true,                 
	numbers=left,                    
	numbersep=5pt,
	showspaces=false,                
	showstringspaces=false,
	showtabs=false,                  
	tabsize=2,
	numberbychapter=false
}
\lstset{style=mystyle}
\renewcommand{\lstlistingname}{Código}

% ---
% Configurações do pacote backref
% Usado sem a opção hyperpageref de backref
\renewcommand{\backrefpagesname}{Citado na(s) página(s):~}
% Texto padrão antes do número das páginas
\renewcommand{\backref}{}
% Define os textos da citação
\renewcommand*{\backrefalt}[4]{
	\ifcase #1 %
		Nenhuma citação no texto.%
	\or
		Citado na página #2.%
	\else
		Citado #1 vezes nas páginas #2.%
	\fi}%
% ---

% ---
% Informações de dados para CAPA e FOLHA DE ROSTO
% ---
\titulo{AutoForm- Sistema para o Registro de Produtos Controlados no SIGMA via Arquivo Eletrônico em Lote (AEL) da Brigada Militar do RS }
\autor{Erik Barcella Trisch}
\local{Osório}
\data{2023}
\orientador{Bruno Chagas Fernandes}
\coorientador{Márcio José de Lemos}

\instituicao{%
  Instituto Federal de Educação, Ciência e Tecnologia do Rio Grande do Sul -- IFRS
  \par
  \textit{Campus} Osório
  \par
  Curso Superior de Tecnologia em Análise e Desenvolvimento de Sistemas}
\tipotrabalho{Trabalho de Conclusão de Curso}
% O preambulo deve conter o tipo do trabalho, o objetivo, 
% o nome da instituição e a área de concentração 
\preambulo{Trabalho de Conclusão de Curso apresentado como requisito parcial para obtenção do título de Tecnólogo em Análise e Desenvolvimento de Sistemas.}
% ---


% ---
% Configurações de aparência do PDF final

% alterando o aspecto da cor azul
\definecolor{blue}{RGB}{41,5,195}

% informações do PDF
\makeatletter
\hypersetup{
     	%pagebackref=true,
		pdftitle={\@title}, 
		pdfauthor={\@author},
    	pdfsubject={\imprimirpreambulo},
	    pdfcreator={LaTeX with abnTeX2},
		pdfkeywords={trabalho de concusão de curso}{ifrs}{campus osório}{ads}{análise e desenvolvimento de sistemas}, %adicionar as keywords do trabalho
		colorlinks=true,       		% false: boxed links; true: colored links
    	linkcolor=blue,          	% color of internal links
    	citecolor=blue,        		% color of links to bibliography
    	filecolor=magenta,      		% color of file links
		urlcolor=blue,
		bookmarksdepth=4
}
\makeatother
% --- 

% --- 
% Espaçamentos entre linhas e parágrafos 
% --- 
% O tamanho do parágrafo é dado por:
\setlength{\parindent}{1.3cm}
% Controle do espaçamento entre um parágrafo e outro:
\setlength{\parskip}{0.2cm}  % tente também \onelineskip

% ---
% compila o indice
% ---
\makeindex
% ---

% ----
% Início do documento
% ----
\begin{document}

% Seleciona o idioma do documento (conforme pacotes do babel)
%\selectlanguage{english}
\selectlanguage{brazil}

% Retira espaço extra obsoleto entre as frases.
\frenchspacing 

% ----------------------------------------------------------
% ELEMENTOS PRÉ-TEXTUAIS
% ----------------------------------------------------------
% \pretextual

% ---
% Capa ((Obrigatório)
% ---
\imprimircapa
% ---

% ---
% Folha de rosto (Obrigatório)
% (o * indica que haverá a ficha bibliográfica)
% ---
\imprimirfolhaderosto*

% Inserir errata (Opcional)
% ---
%\input{elementos-pre-textuais/errata}
% ---

% ---
% Inserir folha de aprovação (Obrigatório)
% ---
% Isto é um exemplo de Folha de aprovação, elemento obrigatório da NBR
% 14724/2011 (seção 4.2.1.3).
%
\begin{folhadeaprovacao}

  \begin{center}
    {\ABNTEXchapterfont\large\imprimirautor}

    \vspace*{\fill}\vspace*{\fill}
    \begin{center}
      \ABNTEXchapterfont\bfseries\Large\imprimirtitulo
    \end{center}
    \vspace*{\fill}
    
    \hspace{.45\textwidth}
    \begin{minipage}{.5\textwidth}
        \imprimirpreambulo
    \end{minipage}%
    \vspace*{\fill}
   \end{center}
     
   %Adicionar o texto abaixo depois do trabalho ser aprovado pela banca.   
   %Trabalho aprovado. \imprimirlocal, xx de xxxxxxxxx de 201X:

   \assinatura{\textbf{\imprimirorientador} \\ Orientador} 
   \assinatura{\textbf{Professor} \\ Convidado 1}
   \assinatura{\textbf{Professor} \\ Convidado 2}
   %\assinatura{\textbf{Professor} \\ Convidado 3}
   %\assinatura{\textbf{Professor} \\ Convidado 4}
    
    \vspace*{1.5cm}
   \begin{center}
    \vspace*{0.5cm}
    {\large\imprimirlocal}
    \par
    {\large\imprimirdata}
    \vspace*{1cm}
  \end{center}
  
\end{folhadeaprovacao}
% ---

% ---
% Dedicatória (Opcional)
% ---
\begin{dedicatoria}
   \vspace*{\fill}
   \centering
   \noindent
   \textit{ Este trabalho é dedicado a...} \vspace*{\fill}
\end{dedicatoria}
% ---

% ---
% Agradecimentos (Opcional)
% ---
\begin{agradecimentos}
    Os agradecimentos principais são destinados à minha esposa Vitória e à minha família, que estiveram sempre ao meu lado, proporcionando apoio e incentivo ao longo desta jornada. Vocês são a minha base, as fontes constantes de inspiração que tornaram possível alcançar este objetivo
    
    Agradeço imensamente ao Sd. Tiago Costa dos Santos por ser o elo de idealização deste trabalho junto à Brigada Militar, fornecendo apoio e suporte essenciais para a conclusão deste projeto. Sua colaboração foi fundamental para a realização deste trabalho.

    Aos meus dedicados orientador e coorientador, Bruno Chagas Alves Fernandes e Márcio José de Lemos, expresso profundos agradecimentos. Suas orientações foram cruciais, e a dedicação, paciência e contribuições de ambos enriqueceram significativamente este trabalho. Agradeço pelo comprometimento e pelos valiosos ensinamentos que foram essenciais para o sucesso deste projeto acadêmico.

    Agradeço também a todos os demais que contribuíram de alguma forma para que isso fosse possível.




\end{agradecimentos}
% ---

% ---
% Epígrafe (Opcional)
% ---
\begin{epigrafe}
    \vspace*{\fill}
    \begin{flushright}
        \textit{''Só se pode alcançar um grande êxito quando nos mantemos fiéis a nós mesmos'' \\
        Friedrich Nietzsche}
    \end{flushright}
\end{epigrafe}

% ---

% ---
% RESUMOS
% ---

% resumo em português (Obrigatório)
\setlength{\absparsep}{18pt} % ajusta o espaçamento dos parágrafos do resumo
\begin{resumo}
	% Em razão da demanda da Brigada Militar do Rio Grande do Sul por um sistema que
	% O Sistema de Gerenciamento Militar de Armas (SIGMA) é um sistema informatizado utilizado
	% pelo Comando do Exército para gerenciar o registro e transferência de armas de fogo no
	% Brasil. 
	A Brigada Militar do Rio Grande do Sul é responsável por gerar um documento eletrônico denominado AEL que inclui dados sobre as armas registradas no estado, e encaminhar à Diretoria de Fiscalização de Produtos Controlados do Exército para cadastro no Sistema de Gerenciamento Militar de Armas (SIGMA). Em razão da demanda apresentada
	pela BM RS por um sistema que sustente a execução deste processo, foi sugerido o desenvolvimento desta aplicação web, denominada AutoForm, desenvolvida com a linguagem JavaScript
	em conjunto com os frameworks React e NodeJS. Utilizando-se de estratégias e metodologias
	que serão abordados durante esta pesquisa, visando facilitar o preenchimento das informações pelo
	operador, otimizar o tempo de execução desta tarefa, aumentar a eficácia, e contemplar todos os
	requisitos necessários para geração do AEL garantindo que este esteja completo e correto antes
	de ser submetido ao SIGMA.

	
	\textbf{Palavras-chave}: Arquivo Eletrônico em Lote, SIGMA, Brigada Militar, Aplicação Web, React, NodeJS,  JavaScript.  %alterar para as palavras-chave do trabalho
\end{resumo}

% resumo em inglês (Obrigatório)
\begin{resumo}[Abstract]
 \begin{otherlanguage*}{english}
   This is the english abstract.

   \vspace{\onelineskip}
 
   \noindent 
   \textbf{Keywords}: latex. abntex. text editoration. %alterar para as palavras-chave do trabalho
 \end{otherlanguage*}
\end{resumo}

% resumo em francês (Opcional)
%\input{elementos-pre-textuais/resume}

% resumo em espanhol (Opcional)
%\input{elementos-pre-textuais/resumen}

% ---

% ---
% inserir lista de ilustrações (Opcional)
% ---
\pdfbookmark[0]{\listfigurename}{lof}
\listoffigures*
\cleardoublepage
% ---

% ---
% inserir lista de tabelas (Opcional)
% ---
\pdfbookmark[0]{\listtablename}{lot}
\listoftables*
\cleardoublepage
% ---

% ---
% inserir lista de abreviaturas e siglas (Opcional)
% ---
\begin{siglas}
  \item[SIGMA] Sistema de Gerenciamento Militar de Armas
  \item[AEL] Arquivos Eletrônicos em Lote 
  \item[SINARM] Sistema Nacional de Armas 
  \item[QG] 
  \item[OM] 
  \item[HTML]
  \item[CSS]  
\end{siglas}
% ---

% ---
% inserir lista de símbolos (Opcional)
% ---
%\input{elementos-pre-textuais/lista-de-simbolos}
% ---

% ---
% inserir o sumario (Obrigatório)
% ---
\pdfbookmark[0]{\contentsname}{toc}
\tableofcontents*
\cleardoublepage
% ---


% ----------------------------------------------------------
% ELEMENTOS TEXTUAIS
% ----------------------------------------------------------
\textual

% ----------------------------------------------------------
% Introdução (Obrigatório)
% ----------------------------------------------------------
% ----------------------------------------------------------
% Introdução (com numeração)
% ----------------------------------------------------------
\chapter{Introdução}
% ----------------------------------------------------------
O Sistema de Gerenciamento Militar de Armas (SIGMA) é um sistema informatizado utilizado
pelo Comando do Exército para gerenciar o registro e transferência de armas de fogo no
Brasil. 
A Brigada Militar do Rio Grande do Sul é reconhecida pelo seu alto nível de profissionalismo e 
adesão às boas práticas disciplinares.
Complementando essas características, a utilização de uma ferramenta especializada tem o potencial de 
aprimorar ainda mais os processos administrativos da corporação.
Ao adotar essa ferramenta, espera-se obter maior eficiência, confiabilidade e alinhamento com as 
exigências do Exército Brasileiro. Embora ainda não tenhamos detalhado a ferramenta específica neste projeto,
é importante destacar que sua implementação visa otimizar os procedimentos administrativos da Brigada Militar, 
proporcionando resultados mais eficazes e alavancando a excelência operacional da instituição.
Atualmente, na Brigada Militar do Rio Grande do Sul, o setor interno é responsável por controlar e 
manter diversos processos administrativos, inclusive interligados a outros órgãos públicos. 
Um dos desafios enfrentados pelos operadores que realizam esta atividade é a digitação manual de dados em um 
arquivo de texto, chamado DataList, seguido pela formatação e adequação do documento ao modelo padrão 
estabelecido pelo Exército Brasileiro, que possui regras específicas de indexação das informações
Para aprimorar a eficiência -nesse processo.
\index{citações!diretas}Utilize o ambiente \texttt{citacao} para incluir
citações diretas com mais de três linhas:
\begin{citacao}
motivação em simplificar e melhorar os fluxos de trabalho, levar a reduzir custos operacionais e implementar novas aplicações mais rápidas através da automação, programação e gestão das transferências de arquivo\cite[5.3]{AndradeJunior}
\end{citacao}


\section{Justificativa}

\section{Objetivos}

\subsection{Objetivos Gerais}

\subsection{Objetivos específicos}

% ---


% ----------------------------------------------------------
% Desenvolvimento (Obrigatório)
% ----------------------------------------------------------
% ---
% Capitulo 1
% ---
%% abtex2-modelo-include-comandos.tex, v-1.9.6 laurocesar
%% Copyright 2012-2016 by abnTeX2 group at http://www.abntex.net.br/ 
%%
%% This work may be distributed and/or modified under the
%% conditions of the LaTeX Project Public License, either version 1.3
%% of this license or (at your option) any later version.
%% The latest version of this license is in
%%   http://www.latex-project.org/lppl.txt
%% and version 1.3 or later is part of all distributions of LaTeX
%% version 2005/12/01 or later.
%%
%% This work has the LPPL maintenance status `maintained'.
%% 
%% The Current Maintainer of this work is the abnTeX2 team, led
%% by Lauro César Araujo. Further information are available on 
%% http://www.abntex.net.br/
%%
%% This work consists of the files abntex2-modelo-include-comandos.tex
%% and abntex2-modelo-img-marca.pdf
%%

% ---
% Este capítulo, utilizado por diferentes exemplos do abnTeX2, ilustra o uso de
% comandos do abnTeX2 e de LaTeX.
% ---
 
\chapter{Referencial Teórico}\label{referecial_teorico}

% % escrever algo entre o inicio de cada capitulo 
% - Ultimo paragrafo da introdução , irei mencionar o abordado em cada capitulo 
% - Inicio de cada seção conter uma descrição 
% - Palavras em ingles italico \texitt
Neste capitulo é abordado o referencial teórico, utilizado como embasamento para a construção deste trabalho.
% ---
\section{ Controle na Segurança Publica Brasileira }
Atualmente no Brasil existem dois sistemas informatizados utilizados por órgãos públicos para realizar a regulamentação e o monitoramento de armas de fogo, munições e demais produtos controlados.
Sendo estes, o Sistema de Gerenciamento Militar de Armas (SIGMA) e o Sistema Nacional de Armas (SINARM). O SIGMA é administrado pelo Exército Brasileiro e é responsável pelo controle de armas de fogo e munições no âmbito da Força. O SINARM é administrado pela Polícia Federal e é responsável pelo controle de armas de fogo e munições em poder da população civil \cite{shotEntendaSinarm}.

\subsection{SIGMA e SINARM}\label{sigmaesinarm}
A principal diferença entre SIGMA e SINARM é o âmbito de atuação. O SIGMA é responsável pelo controle de armas de fogo e munições no âmbito do Exército Brasileiro, enquanto o SINARM é responsável pelo controle de armas de fogo e munições em poder da população civil.
Outra diferença entre os dois sistemas é a natureza das informações que eles gerenciam. O SIGMA gerencia informações sobre armas de fogo e munições de uso militar, enquanto o SINARM gerencia informações sobre armas de fogo e munições de uso civil.
A seguir, está uma tabela comparativa que resume as principais diferenças entre ambos: 
\begin{table}[ht]
	\centering
	\caption{Comparação entre SIGMA e SINARM}
	\begin{tabularx}{\textwidth}{lXp{5cm}}
	  \toprule
	  Característica & SIGMA & SINARM \\
	  \midrule
	  Âmbito de atuação & Exército Brasileiro & População civil \\
	  Natureza das informações & Armas de fogo e munições de uso militar & Armas de fogo e munições de uso civil \\
	  Responsável pela administração & Exército Brasileiro & Polícia Federal \\
	  \bottomrule
	\end{tabularx}
  \end{table}

  
% \section{Segurança Publica - Integração SIGMA e SINARM}
% \cite[Em 2019, a Polícia Federal autorizou que o Exército tenha acesso ao Sinarm, já os militares não permitiram a liberação das informações do Sigma para os agentes da PF]{sbtnewsExxE9rcitoPolxEDcia}
% \cite{inDECRETO9847}

\section{ Sistema SIGMA } 
O Sistema de Gerenciamento Militar de Armas (SIGMA) é um sistema computacional desenvolvido pelo Centro de Desenvolvimento de Sistemas (CDS) do Exército Brasileiro e  implantado em 2003 que vem sendo constantemente atualizado para atender às necessidades da força \cite{fenemeReunixE3oSobre}.

\subsection{Contexto de implantação}
O contexto da implantação do SIGMA foi a necessidade de modernizar o sistema de controle de armas de fogo e munições do Exército Brasileiro.
O SIGMA foi desenvolvido com base nas melhores práticas internacionais de controle de armas de fogo. O sistema é integrado a outros sistemas de informação do Exército Brasileiro, o que permite a troca de dados e informações entre as diferentes áreas da força \cite{fenemeReunixE3oSobre}.

\subsection{AEL}
O Arquivo Eletrônico em Lote (AEL) é um arquivo digital que contém as informações necessárias para o cadastro de produtos controlado no SIGMA.
O AEL é utilizado para o cadastro de armas de fogo de diversas entidades, como as Forças Armadas, as forças auxiliares, a Polícia Militar e o Corpo de Bombeiros \cite{ExércitoBrasileiro}.

O Objetivo do AEL no sistema SIGMA é permitir o cadastro de produtos controlados de diversas entidades de forma centralizada e organizada. O AEL é um elemento importante do SIGMA, pois permite que o Exército Brasileiro tenha um controle mais eficiente das armas de fogo em circulação no país \cite{ExércitoBrasileiro}.

\subsection{Arquivo AEL na Brigada Militar do Rio Grande do Sul}
O AEL no contexto da BM RS, deve ser gerado para o cadastro de armas de fogo de policiais militares. O arquivo deve conter as seguintes informações:
\begin{itemize}
    \item Identificação da Brigada Militar: número do QG, código da OM e nome da OM.
    \item Identificação do armamento: número da arma, tipo de arma, marca, modelo, calibre e série.
    \item Identificação do proprietário: nome completo, CPF, RG, endereço e telefone.
    \item Além das demais informações especificadas nos anexos \ref{sec:anexoA2} e \ref{sec:anexoA3}.
\end{itemize}


O AEL deve ser gerado em um formato texto, seguindo um layout pré-definido e estar conforme os parâmetros de indexação das informações constantes nos anexos \ref{sec:anexoA1} e \ref{sec:anexoA4}.

\section{Gerenciamento de processos}
É uma abordagem disciplinada e sistemática que envolve práticas relacionadas aos processos de negócio, automatizados ou não, com o objetivo de alcançar resultados consistentes e alinhados com as metas estratégicas de uma organização. 
Conforme \citeonline{davila2008inovaccao} ``As organizações tentam inovar para se diferenciar e obter vantagens competitivas, tanto pela melhoria nos bens/serviços fornecidos quanto pela eficiência operativa''.
 Pode-se concluir que os sistemas de informação oferecem inúmeros benefícios para uma organização, sejam eles para melhorar o fluxo de informação, as tomadas de decisões, o controle de qualidade, ou ampliar a produtividade.
 
 \subsection{BPMN}
O modelo e notação de processos de negócios (\textit{Business Process Model and Notation}) é uma notação gráfica padronizada para desenhar processos de negócios em um fluxograma. A diagramação BPMN é intuitiva e permite a representação de detalhes complexos do processo. A simbologia deste modelo serve como uma linguagem padrão, colocando um fim na lacuna de comunicação entre a modelagem do processo e sua execução, para \citeonline{bitencourt2016elicitaccao}:
\begin{citacao}
	
	Modelo de processos de negócio representa os processos de negócio de uma empresa e permite a documentação, simulação, compartilhamento, implementação, avaliação e melhoramento continuo das operações, com o intuito de compreender o funcionamento da organização e os aspectos do seu domínio \cite{bitencourt2016elicitaccao}.
	
\end{citacao}
Em resumo, o levantamento e registro da situação atual dos processos, seguido por uma análise aprofundada, são práticas essenciais para promover a eficiência, a eficácia e a adaptação contínua dentro de uma organização. Essa abordagem sistemática para entender e aprimorar os processos é fundamental para a sustentabilidade organizacional.


\section{Arquitetura Cliente Servidor}
Nessa arquitetura, o software é dividido em duas partes principais: o cliente e o servidor.

O cliente é a parte do sistema que interage diretamente com o usuário. Ele envia solicitações de serviço ao servidor e exibe os resultados recebidos ao usuário. O cliente pode ser um aplicativo de desktop, um aplicativo móvel ou um navegador da web, dependendo do tipo de sistema que está sendo desenvolvido \cite{flanagan2012javascript}.

O servidor é responsável por processar as solicitações recebidas do cliente e fornecer os recursos ou serviços solicitados. Ele possui os recursos necessários para atender às solicitações, como bancos de dados, aplicativos e serviços web. O servidor está sempre ativo, aguardando solicitações dos clientes e respondendo a elas de maneira apropriada \cite{arqClientServer2}.

A comunicação entre o cliente e o servidor ocorre por meio de uma rede, geralmente a Internet. O cliente envia uma solicitação para o servidor, especificando o tipo de serviço desejado e quaisquer parâmetros necessários.
Então de acordo com \citeonline{arqClientServer2}.
\begin{citacao}
O servidor, quando recebe a mensagem, extrai os parâmetros 
e chama o procedimento especificado na mensagem. No fim da execução do procedimento é 
realizada a operação inversa, colocando os resultados e enviando a mensagem de resposta ao 
processo cliente \cite{arqClientServer2}.
\end{citacao}

    Logo uma das principais vantagens da arquitetura cliente-servidor é a divisão clara de responsabilidades entre ambos. 
O cliente lida com a interface do usuário e a apresentação dos dados, enquanto o servidor cuida do processamento das solicitações e do acesso aos recursos. Isso permite uma melhor organização do sistema e facilita a manutenção e a escalabilidade.
Além disso, a arquitetura cliente-servidor permite que vários clientes acessem o mesmo servidor simultaneamente. Isso possibilita o compartilhamento de recursos e serviços, o que é especialmente útil em ambientes corporativos \cite{arqClientServer2}.

A \autoref{fig:grafico-client-server} demonstra a maneira como ocorre essa comunicação:

\begin{figure}[htb]
    \caption{\label{fig:grafico-client-server}Arquitetura Cliente-Servidor}
    \begin{center}
        \includegraphics[scale=0.9]{imagens/arquitetura-cliente-servidor.png}
    \end{center}
    \legend{Fonte: \citeonline{Redes}}
\end{figure}


\section{Arquitetura MVC}
A arquitetura MVC \textit{Model-View-Controller} é um padrão de design que organiza o código de uma aplicação em três componentes principais: \textit{Model} (Modelo), \textit{View} (Visão) e \textit{Controller} (Controlador). Cada componente tem uma responsabilidade específica na aplicação, o que ajuda a manter o código modular, escalável e de fácil manutenção \cite{engsoftmoderna}.

\begin{itemize}
    \item Visão: Lida com a apresentação dos dados ao usuário e interage com o Modelo. A Visão exibe as informações e envia eventos do usuário para o Controlador.
    \item Controlador: Recebe entradas do usuário, processa essas entradas (geralmente envolvendo o Modelo) e atualiza a Visão. O Controlador age como um intermediário entre o Modelo e a Visão.
    \item Modelo: Representa a lógica de negócios e os dados da aplicação. Geralmente, o modelo é responsável pela interação com o banco de dados e pela manipulação dos dados.
\end{itemize}
Conforme exemplificado o fluxo da arquitetura MVC na \autoref{fig:grafico-mvc}

\begin{figure}[htb]
    \caption{\label{fig:grafico-mvc}Arquitetura MVC}
    \begin{center}
        \includegraphics[scale=0.7]{imagens/arquitetura-mvc.png}
    \end{center}
    \legend{Fonte: \citeonline{engsoftmoderna}}
\end{figure}


\section{Aplicações Web}
Aplicações web são programas que são executados em navegadores e são acessados por meio de internet.
Surgiram na década de 1990 e se tornaram populares por permitirem a interação do usuário e o processamento de dados.

Existem dois tipos principais: estáticas (HTML, CSS e JavaScript) e dinâmicas (linguagens de programação do lado do servidor).
De acordo com \citeonline{aplicacoesWeb} as aplicações estáticas geralmente consistem em páginas web com conteúdo fixo, sem interação avançada ou processamento de dados em tempo real. O navegador do cliente solicita páginas estáticas ao servidor, que retorna arquivos HTML, CSS e JavaScript. A renderização e interação ocorrem no navegador.

Aplicações Dinâmicas apresentam interatividade avançada e processamento de dados em tempo real quando o navegador solicita uma página ao servidor. O servidor executa a lógica de negócios, acessa dados do banco de dados, gera dinamicamente o conteúdo HTML e o envia de volta ao navegador. Pode haver interações adicionais entre o navegador e o servidor 
oferecendo diversos benefícios como acessibilidade, atualização, redução de custos e escalabilidade \cite{aplicacoesWeb}.

\subsection{Linguagem JavaScript}
Uma linguagem de programação amplamente usada no desenvolvimento web de acordo com a organização \citeonline{mozillaJavaScript}. Ela permite adicionar interatividade e dinamismo a páginas da web. Além de ser usado no desenvolvimento de interfaces de usuário, o JavaScript também pode ser usado no desenvolvimento de aplicativos do lado do servidor (backend) com o uso de tecnologias como o Node.js. Conforme \citeonline{flanagan2012javascript} ``Javascript já deixou para trás suas raízes como linguagem de script há muito tempo, tornando-se uma linguagem de uso geral, robusta e eficiente''.

\subsection{Frameworks e Bibliotecas}
A necessidade da utilização de \textit{frameworks} surgiu com a complexidade crescente das aplicações de software.
\begin{citacao}
	Diversos \textit{frameworks} têm sido desenvolvidos nas duas últimas décadas, visando o
	reuso de software e consequentemente a melhoria da produtividade, qualidade e
	manutenibilidade \cite{maldonado2002padroes}.
\end{citacao}
Sendo estes estruturas ou conjuntos de ferramentas que fornecem uma base organizada para o desenvolvimento de aplicações. Eles oferecem uma estrutura pré-definida que acelera o processo de desenvolvimento, promove a reutilização de código e estabelece padrões de boas práticas. No contexto do desenvolvimento de software, os frameworks desempenham um papel significativo, influenciando a forma como as aplicações são projetadas, implementadas e mantidas \cite{maldonado2002padroes}.

Portanto para \citeonline[p.~23]{maldonado2002padroes} os \textit{frameworks} estão divido em dois tipos sendo eles:
\begin{itemize}
    \item Caixa preta: Uma abordagem caixa preta trata um sistema ou componente como uma entidade onde o foco está no comportamento externo, sem conhecimento detalhado de sua implementação interna, pois na maioria dos casos o desenvolvedor não tem acesso ao código fonte e a ênfase está nos resultados visíveis e nas funcionalidades oferecidas pelo sistema, sem a necessidade de compreender a lógica interna do sistema \cite[p.~23]{maldonado2002padroes}.
    \item Caixa branca: Em contraste, uma abordagem caixa branca envolve uma compreensão detalhada da implementação interna de um sistema ou componente, incluindo sua lógica, estrutura e fluxo de controle e a ênfase está na compreensão completa do sistema, possibilitando otimizações, depuração precisa e ajustes finos \cite[p.~23]{maldonado2002padroes}.
\end{itemize}

Entretanto muito semelhante aos \textit{frameworks} exceto por algumas diferenças abordados na \autoref{tab:comparacao_bibliotecas_frameworks}, existem também as bibliotecas de \textit{software} que são conjuntos de códigos preexistentes e funcionalidades encapsuladas que foram desenvolvidos para abordar tarefas comuns ao desenvolvimento de sistemas. 
Conforme \citeonline{cechinel2017avaliaccao} torna-se evidente que esses recursos desempenham um papel fundamental na eficiência e na evolução contínua do desenvolvimento de \textit{software}. Desde a reutilização inteligente de código até a adaptação às inovações tecnológicas.


\begin{table}[h]
    \centering
    \begin{tabularx}{\textwidth}{X X X}
      \toprule
      \textbf{Característica} & \textbf{Bibliotecas} & \textbf{Frameworks} \\
      \midrule
      Definição & Fornecem funcionalidades específicas. & Oferecem uma estrutura abrangente. \\
      \midrule
      Flexibilidade & Mais flexibilidade para os desenvolvedores. & Menos flexibilidade devido a maiores convenções. \\
      \midrule
      Integração & Integração opcional e modular. & Impõem uma estrutura mais integrada. \\
      \midrule
      Modularidade & Modular, escolha de partes específicas. & Estrutura monolítica, menos modularidade. \\
      \bottomrule
    \end{tabularx}
    \caption{Comparação Entre Bibliotecas e Frameworks no Desenvolvimento de Software}
    \label{tab:comparacao_bibliotecas_frameworks}
  \end{table}

\subsubsection{Node}
 Conforme o site oficial do \citeonline{Nodejs}, este é um ambiente de tempo de execução javascript que permite que o javascript seja executado no lado do servidor. Utilizando o mecanismo V8 do Google Chrome para executar código javascript fora do navegador.
 Com o Node.js, é possível criar aplicativos web e serviços \textit{backend} usando javascript. Ele fornece uma variedade de recursos e uma ampla gama de bibliotecas e frameworks, tornando-o uma escolha popular para o desenvolvimento de servidores e APIs \cite{Nodejs}.
Portanto para \citeonline{pereira2014aplicações}.
    \begin{citacao}
        Node.js é multiprotocolo, ou seja, com ele será possível trabalhar com os protocolos: HTTP, HTTPS, FTP, SSH, DNS, TCP, UDP, WebSockets e também existem outros.Toda aplicação web necessita de um servidor para disponibilizar todos os seus  recursos \cite{pereira2014aplicações}.
    \end{citacao}

Amplamente utilizado em conjunto com o node, é o framework express para aplicativos web do lado do servidor, construído sob a base nativa HTTP do Node.js.
     Ele fornece uma abordagem simplificada para lidar com solicitações HTTP, roteamento e manipulação de middleware. 
	 O Express permite criar facilmente APIs robustas e eficientes, tornando o desenvolvimento de aplicativos web mais rápido e produtivo. É um dos frameworks mais populares para o desenvolvimento de servidores com Node.js
     \cite{pereira2014aplicações}.

\subsubsection{ReactJS}
React é uma biblioteca javascript \textit{Open Source} lançado em 2013 pela \citeonline{ReactMet54}, que rapidamente ganhou popularidade devido a sua abordagem inovadora utilizada para criar interfaces de usuário. 
Através da possibilidade de escrever código utilizando a sintaxe JSX, a qual é uma convenção opcional no react que possibilita ter a organização do código javascript de maneira mais próxima à estrutura de marcação XML ou HTML.

    Logo essa biblioteca permite desenvolver componentes reutilizáveis e interativos utilizados na construção de \textit{UIS} modernas e responsivas. Possibilitando o desenvolvimento de aplicações web complexas e dinâmicas, dividindo-as em componentes reutilizáveis \citeonline{ReactMet54}.

    A partir da premissa de que a abordagem principal do React é baseada em componentes, isso facilita a criação e o gerenciamento de estado dos elementos da interface. Permitindo a criação de aplicações eficientes e escaláveis, sendo como principal característica a renderização reativa, o que faz com que a interface do usuário seja atualizada automaticamente quando o estado dos dados é alterado. Isso simplifica o desenvolvimento e melhora o desempenho, pois apenas as partes afetadas da interface são atualizadas, em vez de recarregar a página inteira \cite{ReactMet54}.

\section{Banco de dados NOSQL}
Banco de dados NoSQL é um tipo de banco de dados que difere dos bancos de dados relacionais tradicionais (SQL) em sua estrutura de armazenamento e modelo de dados. NoSQL significa \textit{Not Only SQL} (Não Apenas SQL) e abrange diversos tipos de bancos de dados que oferecem uma abordagem alternativa para o armazenamento e recuperação de dados.
Para \citeonline{pereira2014aplicações} uma das vantagens em se trabalhar com um banco de dados desse modelo é o grande suporte oferecido pela comunidade do nodejs e uma vasta gama de compatibilidade com diversas tecnologias.

Dentre as principais tecnologias que tem uma alta sinergia nesse padrão não relacional são: 


\begin{itemize}
    \item MongoDB: um eficiente e popular banco de dados NoSQL. Que utiliza um sistema de gerenciamento de banco de dados orientado a documentos, o que significa que os dados são armazenados em documentos semelhantes a JSON, em vez de tabelas com linhas e colunas como em um banco de dados relacional \cite{pereira2014aplicações}.
    Outra característica importante do MongoDB é sua capacidade de escalar horizontalmente. Oferecendo recursos avançados, como indexação, consultas poderosas e suporte a transações, tornando-o adequado para uma ampla gama de aplicações. É frequentemente utilizado em aplicativos web, análise de dados, e outras aplicações que exigem flexibilidade e escalabilidade \cite{pereira2014aplicações}.
    \item Mongoose:  Uma biblioteca ODM \textit{Object Data Modeling} para Node.js e MongoDB. Sendo inserida como uma camada de abstração facilitando a conexão, a modelagem de dados, a execução de consultas, e a interação com o banco de dados de maneira eficiente e organizada \cite{pereira2014aplicações}.

\end{itemize}


% \section{Backend}
% É a parte de um sistema ou aplicação que lida com a lógica de negócios, processamento de dados e a comunicação com o banco de dados. Envolve a criação de servidores, APIs (Application Programming Interfaces) e serviços que fornecem os dados e funcionalidades necessárias para o funcionamento do sistema. Para o desenvolvimento backend, são utilizadas diversas tecnologias, como linguagens de programação (como JavaScript, Python, Java, etc.), bancos de dados (como MySQL, PostgreSQL, MongoDB, etc.) e frameworks (como Node.js, Django, Ruby on Rails, etc.). Então o backend deve ser capaz de servir ao front-end a comunicação em tempo real entre cliente e servidor — que seja rápido, atenda muitos usuários ao mesmo tempo e utilize recursos de I/O (dispositivos de entrada ou saída) de forma eficiente ( RIBEIRO, CAIO , 2013)

% \subsection{Apis Restful}

\subsection{Segurança e autenticação}
A segurança e a autenticação em aplicações web são fundamentais para proteger dados e usuários. Utilizando criptografia,  e práticas de desenvolvimento seguro, é possível mitigar riscos, garantindo a integridade e confiabilidade do sistema. Estratégias como autenticação por token e hashing de senhas fortalecem a proteção, assegurando uma navegação online para o usuário segura e confiável \cite{segurancaeauth}.

\begin{itemize}
    \item Passport-Local: De acordo com o site oficial do \cite{passport84}, este é uma estratégia de autenticação fornecida pelo Passport.js para autenticar usuários usando um nome de usuário e senha em aplicativos Node.js. Ele é facilmente integrado a qualquer aplicativo ou framework que suporte middlewares do estilo \textit{Connect}, incluindo o Express. O Passport-local requer um retorno de chamada de verificação que valida as credenciais do usuário. Ele pode ser configurado para realizar a autenticação localmente, verificando o nome de usuário e a senha no banco de dados da aplicação. 
    \item Token JWT: \textit{JSON Web Token} (JWT) fornece uma abordagem segura para a troca de informações entre cliente e servidor por meio de um token gerado o qual tem como resultado final um objeto JSON, conforme explicado por Lucas, Ana e Jackson.
    \begin{citacao}
        O token gerado pelo JWT é salvo
        no dispositivo do usuário e suas informações podem ser verificadas a cada solicitação,
        pois são criptografadas utilizando um segredo, através do algoritmo \textit{HMAC} ou de um par de chaves públicas e privadas, garantindo assim a sua confiabilidade \cite{segurancaeauth}.
    \end{citacao}
    A \autoref{fig:grafico-jwt} exemplifica o fluxo de autenticação por token jwt: 

\begin{figure}
    \caption{\label{fig:grafico-jwt} Fluxo de autenticação JWT}
    \begin{center}
        \includegraphics[scale=0.9]{imagens/jwt.png}
    \end{center}
    \legend{Fonte: \citeonline{segurancaeauth}}
\end{figure}
    
\end{itemize}

% \section{Frontend}
% É a parte de um sistema ou aplicação que os usuários interagem diretamente. Envolve a criação da interface do usuário, a implementação de elementos visuais, como layout, design, botões, formulários, etc., e a interação com o usuário por meio de eventos e ações. Para o desenvolvimento front-end, são utilizadas tecnologias como HTML (Hypertext Markup Language), CSS (Cascading Style Sheets) e JavaScript. Todo o HTML e o CSS que escrevemos ganha vida dentro dos navegadores utilizados por quem acessa nossas páginas e sites (MAZZA LUCAS, 2012)

% \begin{itemize}
%     \item HTML: (HyperText Markup Language) é a linguagem de marcação usada para estruturar e exibir o conteúdo de uma página da web. Ele fornece uma estrutura básica para a criação de elementos, como cabeçalhos, parágrafos, listas, links e imagens. O HTML é a espinha dorsal de qualquer página da web e é complementado por CSS e JavaScript para fornecer estilos e interatividade.
%     \item CSS: CSS (Cascading Style Sheets) é uma linguagem usada para estilizar a aparência dos elementos em uma página da web. Ele permite controlar cores, fontes, margens, posicionamento e outros aspectos visuais dos elementos HTML. O CSS é usado em conjunto com o HTML para criar layouts atraentes e responsivos. Ele oferece flexibilidade para personalizar o estilo de um site e torná-lo visualmente agradável para os usuários. 


% \end{itemize}
 


% ---
% ---
% Capitulo 2
% ---
\chapter{Trabalhos Relacionados}
% ---
% ---
\section{TAF- teste de aptidão física da brigada militar do rio grande do sul}
Um estudo feito por gabriela machado durante o curso de Analise e desenvolvimento de sistemas no IFRS- campus Osório em 2018, o taf é uma avaliação física que visa avaliar a aptidão física dos candidatos a ingressar na brigada militar do rio grande do sul. É uma etapa importante do processo seletivo, e tem como objetivo verificar se os candidatos possuem as condições físicas mínimas exigidas para desempenhar as atividades do cargo. Se concentrando em diferentes aspectos, como a validade e a confiabilidade do teste, a relação entre os resultados do taf e o desempenho dos candidatos nas atividades militares, os fatores que influenciam o desempenho dos candidatos no taf, entre outros.

\section{Melhoria de processo pelo BPM, aplicação no setor publico}
O artigo de claudio josé muller e isadora cidade mariano apresenta um relato de uma aplicação da metodologia bpm (business process management), que foi realizada em quatro etapas: (i) planejamento das atividades do bpm; (ii) mapeamento do processo escolhido; (iii) proposta de melhorias e comparação entre o processo atual e o proposto. A metodologia foi adaptada para o contexto de uma organização pública e esta abordagem foi utilizada para modernizar o processo de controle de trânsito animal no brasil. A partir da análise do processo atual foram propostas melhorias a fim de otimizar recursos, melhorar a confiabilidade e aumentar a satisfação de clientes.

\section{Automação de processos manuais}
Por Luiz Roberto de Andrade Júnior as instituições financeiras possuem uma longa história de solução de problemas na área de informática através da criação de novas ferramentas e tecnologias de automação de processos de trabalho, para garantir uma entrega mais rápida de suas tarefas. O presente trabalho apresenta a automação de processos hoje executados manualmente e com a ajuda da ferramenta CONTROL-M desenvolvida pela empresa BMC software. Foi adquirida uma ferramenta capaz de verificar o estado de execução de tarefas agendadas, com a análise de seus resultados gerados. A verificação é realizada por meio de critérios de validação customizáveis pelo usuário da ferramenta em questão
% ---

% ---

% ---
% Capitulo 3
% ---
\chapter{Metodologia}
% ---
% ---
Neste segmento, serão delineadas informações relacionadas à metodologia utilizada para elaboração da aplicação proposta neste projeto. 
Sendo utilizada como base a metodologia de desenvolvimento de software iterativa e incremental.
Este modelo enfatiza ciclos repetitivos curtos, chamados iterações, para construir o sistema incrementalmente. Cada iteração inclui planejamento, design, implementação e testes, permitindo feedback contínuo. 
Prioriza a adaptabilidade a mudanças, flexibilidade e entrega de funcionalidades incrementais, proporcionando entregas intermediárias e valor ao cliente ao longo do processo \cite{engenhariasw}.
Exemplos incluem metodologias ágeis como Scrum e Extreme Programming (XP). 

Discutiremos também na estruturação do desenvolvimento, a análise de requisitos, abrangendo tanto os funcionais quanto os não funcionais.

Na delineação da Estruturação do desenvolvimento, é vital compreender a sinergia entre os componentes, visando otimizar a eficiência e a coesão do sistema em questão. Os requisitos funcionais, sendo pilares fundamentais, demandam uma abordagem analítica aprimorada para assegurar sua implementação fluída.

Quanto aos requisitos não funcionais, devemos explorar suas nuances de maneira minudente, reconhecendo a importância intrínseca de cada faceta no contexto global do projeto. Esta abordagem detalhada proporciona uma compreensão mais profunda, permitindo a delimitação precisa de parâmetros que transcendem as fronteiras meramente operacionais.

\section{Estruturação do desenvolvimento}\label{sec:estruturacao-desenvolvimento}
    A forma como foi organizado o desenvolvimento do presente trabalho consiste inicialmente em reuniões realizadas através de entrevistas conduzidas com um membro da Brigada Militar, SD. Tiago Costa, por vezes juntamente com o orientador ou coorientador no \textit{campus} Osório. Durante essas entrevistas, foram formuladas perguntas pertinentes ao procedimento de geração do AEL.

A coleta de informações foi complementada por meio de análise documental, sendo como principal embasamento a Portaria 136 de 8 de novembro de 2019 do Exército Brasileiro, mais especificamente o anexo D do \citeonline{ExércitoBrasileiro} constante nos Anexos (\ref{sec:anexoA1}, \ref{sec:anexoA2}, \ref{sec:anexoA3}, \ref{sec:anexoA4}). Esta documentação conceitua todas as fases envolvidas no processo de geração do AEL da Brigada Militar do Rio Grande do Sul.

A partir da análise de escopo realizada, planejou-se o uso do modelo de desenvolvimento iterativo neste projeto pois de acordo com \citeonline{engenhariasw}, se um processo interno de desenvolvimento iterativo é usado, o documento de requisitos pode ser muito menos detalhado e quaisquer ambiguidades podem ser resolvidas durante o desenvolvimento do sistema.

Logo a complexidade do procedimento de geração do AEL demanda uma abordagem flexível, permitindo ajustes contínuos à medida que novos pontos de vista são revelados durante entrevistas com SD. Tiago Costa. Essa metodologia proporciona validação incremental das fases, adaptabilidade a mudanças nos requisitos e envolvimento constante, garantindo um desenvolvimento mais preciso.


\section{Análise de Requisitos}
Os requisitos inerentes a aplicação web, foram levantados com base nas informações coletadas durantes as reuniões com os envolvidos conforme citado na \autoref{sec:estruturacao-desenvolvimento} sendo avaliado como as funcionalidades essenciais aquelas que impactam diretamente o pleno funcionamento da aplicação, quais devem ser implementadas durante o desenvolvimento do sistema. Ademais, foi realizada uma avaliação para determinar as funcionalidades desejáveis.

Sendo divida esta em subseções contendo os requisitos funcionais e não funcionais.

\subsection{Requisitos Funcionais}
Nesta subseção, será abordado os requisitos funcionais, aqueles que abrangem e descrevem todas as funcionalidades previstas para o sistema. 

\begin{table}[H]\label{tab:rf01}
    \caption{Requisito Funcional 1}
    \centering
    \includegraphics[scale=0.73]{imagens/rf01.png}
    \legend{Fonte: Autor}
\end{table}
\begin{table}[H]\label{tab:rf02}
    \caption{Requisito Funcional 2}
    \centering
    \includegraphics[scale=0.8]{imagens/rf02.png}
    \legend{Fonte: Autor}
\end{table}
\begin{table}[H]\label{tab:rf03}
    \caption{Requisito Funcional 3}
    \centering
    \includegraphics[scale=0.73]{imagens/rf03.png}
    \legend{Fonte: Autor}
\end{table}
\begin{table}[H]\label{tab:rf04}
    \caption{Requisito Funcional 4}
    \centering
    \includegraphics[scale=0.73]{imagens/rf04.png}
    \legend{Fonte: Autor}
\end{table}
\begin{table}[H]\label{tab:rf05}
    \caption{Requisito Funcional 5}
    \centering
    \includegraphics[scale=0.8]{imagens/rf05.png}
    \legend{Fonte: Autor}
\end{table}
\begin{table}[H]
    \caption{Requisito Funcional 6}\label{tab:rf06}
    \centering
    \includegraphics[scale=0.8]{imagens/rf06.png}
    \legend{Fonte: Autor}
\end{table}
\begin{table}[H]
    \caption{Requisito Funcional 7}\label{tab:rf07}
    \centering
    \includegraphics[scale=0.8]{imagens/rf07.png}
    \legend{Fonte: Autor}
\end{table}
\begin{table}[H]
    \caption{Requisito Funcional 8}\label{tab:rf08}
    \centering
    \includegraphics[scale=0.8]{imagens/rf08.png}
    \legend{Fonte: Autor}
\end{table}
\begin{table}[H]
    \caption{Requisito Funcional 9}\label{tab:rf09}
    \centering
    \includegraphics[scale=0.8]{imagens/rf09.png}
    \legend{Fonte: Autor}
\end{table}
\begin{table}[H]
    \caption{Requisito Funcional 10}\label{tab:rf10}
    \centering
    \includegraphics[scale=0.8]{imagens/rf10.png}
    \legend{Fonte: Autor}
\end{table}
\begin{table}[H]
    \caption{Requisito Funcional 11}\label{tab:rf11}
    \centering
    \includegraphics[scale=0.8]{imagens/rf11.png}
    \legend{Fonte: Autor}
\end{table}
\begin{table}[H]
    \caption{Requisito Funcional 12}\label{tab:rf12}
    \centering
    \includegraphics[scale=0.8]{imagens/rf12.png}
    \legend{Fonte: Autor}
\end{table}
\begin{table}[H]
    \caption{Requisito Funcional 13}\label{tab:rf13}
    \centering
    \includegraphics[scale=0.8]{imagens/rf13.png}
    \legend{Fonte: Autor}
\end{table}
\begin{table}[H]
    \caption{Requisito Funcional 14}\label{tab:rf14}
    \centering
    \includegraphics[scale=0.8]{imagens/rf14.png}
    \legend{Fonte: Autor}
\end{table}



\subsection{Requisitos Não Funcionais}
Na presente subseção, serão apresentados os requisitos não funcionais da aplicação, os quais descrevem as características globais do sistema, indo além das funcionalidades específicas.
E assegurando uma compreensão completa e abrangente dos métodos empregados pela aplicação para alcançar o resultado almejado.

\begin{table}[H]
    \caption{Requisito Não Funcional 1}\label{tab:rnf1}
    \centering
    \includegraphics[scale=0.8]{imagens/rnf1.png}
    \legend{Fonte: Autor}
\end{table}
\begin{table}[H]
    \caption{Requisito Não Funcional 2}\label{tab:rnf2}
    \centering
    \includegraphics[scale=0.8]{imagens/rnf2.png}
    \legend{Fonte: Autor}
\end{table}
\begin{table}[H]
    \caption{Requisito Não Funcional 3}\label{tab:rnf3}
    \centering
    \includegraphics[scale=0.8]{imagens/rnf3.png}
    \legend{Fonte: Autor}
\end{table}
\begin{table}[H]
    \caption{Requisito Não Funcional 4}\label{tab:rnf4}
    \centering
    \includegraphics[scale=0.73]{imagens/rnf4.png}
    \legend{Fonte: Autor}
\end{table}
\begin{table}[H]
    \caption{Requisito Não Funcional 5}\label{tab:rnf5}
    \centering
    \includegraphics[scale=0.8]{imagens/rnf5.png}
    \legend{Fonte: Autor}
\end{table}
\begin{table}[H]
    \caption{Requisito Não Funcional 6}\label{tab:rnf6}
    \centering
    \includegraphics[scale=0.8]{imagens/rnf6.png}
    \legend{Fonte: Autor}
\end{table}
\begin{table}[H]
    \caption{Requisito Não Funcional 7}\label{tab:rnf6}
    \centering
    \includegraphics[scale=0.8]{imagens/rnf7.png}
    \legend{Fonte: Autor}
\end{table}
% % ---
% \section{Tabelas}
% % ---

% \index{tabelas}A \autoref{tab-nivinv} é um exemplo de tabela construída em
% \LaTeX.

% \begin{table}[htb]
% \ABNTEXfontereduzida
% \caption[Níveis de investigação]{Níveis de investigação.}
% \label{tab-nivinv}
% \begin{tabular}{p{2.6cm}|p{6.0cm}|p{2.25cm}|p{3.40cm}}
%   %\hline
%    \textbf{Nível de Investigação} & \textbf{Insumos}  & \textbf{Sistemas de Investigação}  & \textbf{Produtos}  \\
%     \hline
%     Meta-nível & Filosofia\index{filosofia} da Ciência  & Epistemologia &
%     Paradigma  \\
%     \hline
%     Nível do objeto & Paradigmas do metanível e evidências do nível inferior &
%     Ciência  & Teorias e modelos \\
%     \hline
%     Nível inferior & Modelos e métodos do nível do objeto e problemas do nível inferior & Prática & Solução de problemas  \\
%    % \hline
% \end{tabular}
% \legend{Fonte: \citeonline{van86}}
% \end{table}

% Já a \autoref{tabela-ibge} apresenta uma tabela criada conforme o padrão do
% \citeonline{ibge1993} requerido pelas normas da ABNT para documentos técnicos e
% acadêmicos.

% \begin{table}[htb]
% \IBGEtab{%
%   \caption{Um Exemplo de tabela alinhada que pode ser longa
%   ou curta, conforme padrão IBGE.}%
%   \label{tabela-ibge}
% }{%
%   \begin{tabular}{ccc}
%   \toprule
%    Nome & Nascimento & Documento \\
%   \midrule \midrule
%    Maria da Silva & 11/11/1111 & 111.111.111-11 \\
%   \midrule 
%    João Souza & 11/11/2111 & 211.111.111-11 \\
%   \midrule 
%    Laura Vicuña & 05/04/1891 & 3111.111.111-11 \\
%   \bottomrule
% \end{tabular}%
% }{%
%   \fonte{Produzido pelos autores.}%
%   \nota{Esta é uma nota, que diz que os dados são baseados na
%   regressão linear.}%
%   \nota[Anotações]{Uma anotação adicional, que pode ser seguida de várias
%   outras.}%
%   }
% \end{table}
% ---



\chapter{Desenvolvimento da aplicação AutoForm}
% ---
Neste capitulo é apresentado o desenvolvimento do projeto, onde são apresentados,arquitetura do sistema,o processo de produção, e a implementação do sistema.

\subsection{Processo de produção}
O projeto teve sua fase inicial, após o 1º encontro com o cliente Tiago Costa e o coorientador Márcio Lemos no \textit{campus} Osório do IFRS, conforme demonstrado a necessidade pela BMRS de uma aplicação que pudesse ser utilizada para auxiliar no processo de registro do AEL no SIGMA, a fim de agilizar o processo de preenchimento e automatizar a geração do arquivo, bem como resolver outras dificuldades encontradas pelos operadores.
Após a definição de contexto do AEL por parte da BM RS, foi realizado um estudo sobre o assunto, para que fosse possível entender o problema e propor uma solução viável e eficaz. 

Com a elicitação dos requisitos concluída, ocorreu a prototipação das telas da aplicação, utilizando a ferramenta de \textit{design de interface } de usuário \cite{figma}, a fim de validar se o visual e o fluxo da aplicação iriam atender o proposito da instituição.

\subsubsection{Tecnologias}\label{sec:tecnologias}
Após a validação do protótipo, foi realizada a escolha das tecnologias que seriam utilizadas no desenvolvimento da aplicação, sendo elas o \textit{framework} \cite{React22:online}, para o desenvolvimento \textit{frontend}, e o \textit{framework} NodeJS \cite{Nodejs} para o desenvolvimento do \textit{backend} da aplicação web, juntamente com o banco de dados não relacional \cite{MongoDBA45:online}, para o armazenamento dos dados.

Com as tecnologias definidas, foi realizado o desenvolvimento da aplicação web e hospedado em um servidor, para que fosse possível realizar os testes e validações com o cliente, e assim, realizar as correções necessárias. 

Portanto após a primeira versão da aplicação estar disponível na \textit{internet} as adaptações necessárias e atualizações eram organizadas e relatadas mediante troca de mensagens com o cliente.

Para permitir que fosse possível realizar atualizações em produção na aplicação foi utilizada a ferramente\cite{git}, juntamente com a plataforma de hospedagem de código fonte \cite{github}, para que fosse possível realizar o controle de versões e o versionamento do código fonte da aplicação, garantindo sempre uma versão estável e disponivel em produção, enquanto era viável desenvolver paralelamente novas funcionalidades e valida-las com o cliente.

\subsection{Arquitetura da aplicação}
Nesta seção será abordado a visão de como foi projetado a estrutura do sistema que é composto por uma aplicação web, e uma API.

A Arquitetura \textit{MVC} foi a escolhida para a estruturação do sistema, ficando então dispostas as tecnologias mencionadas na \autoref{sec:tecnologias} da seguinte maneira ilustradas pela  \autoref{fig:diagramstacks}.

\begin{figure}[htb]
    \caption{\label{fig:diagramstacks}Disposição das tecnologias na arquitetura MVC}
    \begin{center}
        \includegraphics[scale=0.5]{imagens/diagrama.png}
    \end{center}
    \legend{Fonte: Autor}
\end{figure}


Toda comunicação entre a aplicação web e a API é realizada através de requisições HTTP, onde a aplicação web realiza requisições para a API, e a API responde com os dados solicitados, caso necessário a API busca as informações no banco de dados e retorna para a aplicação web.
As requisições e respostas seguem o padrão REST, onde cada requisição possui um método HTTP, e um \textit{endpoint} que é a URL que identifica o recurso que está sendo solicitado, e a resposta da requisição é um JSON, que contém os dados solicitados.



\subsection{Aplicação Web AutoForm}

\begin{figure}[htb]
    \caption{\label{fig:tela-cadastro}Autoform - Pagina de login}
    \begin{center}
        \includegraphics[scale=0.5]{imagens/login-autoform.png}
    \end{center}
    \legend{Fonte: Autor}
\end{figure}

\begin{figure}[htb]
    \caption{\label{fig:tela-registro}Autoform - Pagina de registros}
    \begin{center}
        \includegraphics[scale=0.6]{imagens/registro-autoform.png}   
     \end{center}
    \legend{Fonte: Autor}
\end{figure}

\begin{figure}[htb]
    \caption{\label{fig:tela-home}Autoform - Pagina inicial home}
    \begin{center}
        \includegraphics[scale=0.5]{imagens/home-autoform.png}
    \end{center}
    \legend{Fonte: Autor}
\end{figure}

\begin{figure}[htb]
    \caption{\label{fig:tela-ael1}Autoform - Pagina criação AEL}
    \begin{center}
        \includegraphics[scale=0.5]{imagens/autoform-ael-gerar.png}
    \end{center}
    \legend{Fonte: Autor}
\end{figure}

\begin{figure}[htb]
    \caption{\label{fig:tela-ael1}Autoform - Pagina criação AEL com opção para selecionar}
    \begin{center}
        \includegraphics[scale=0.5]{imagens/autoform-ael-selecao.png}
    \end{center}
    \legend{Fonte: Autor}
\end{figure}

\begin{figure}[htb]
    \caption{\label{fig:tela-ael2}Autoform - Pagina criação AEL -2}
    \begin{center}
        \includegraphics[scale=0.5]{imagens/autoform-ael-gerar2.png}
    \end{center}
    \legend{Fonte: Autor}
\end{figure}
\begin{figure}[htb]
    \caption{\label{fig:tela-ael3}Autoform - Pagina criação AEL-3}
    \begin{center}
        \includegraphics[scale=0.5]{imagens/autoform-ael-gerar3.png}
    \end{center}
    \legend{Fonte: Autor}
\end{figure}
\begin{figure}[htb]
    \caption{\label{fig:tela-ael4}Autoform - Pagina criação AEL-4}
    \begin{center}
        \includegraphics[scale=0.5]{imagens/autoform-ael-gerar4.png}
    \end{center}
    \legend{Fonte: Autor}
\end{figure}
\begin{figure}[htb]
    \caption{\label{fig:tela-ael-gerado}Autoform - AEL gerado}
    \begin{center}
        \includegraphics[scale=0.5]{imagens/autoform-ael-gerado.png}
    \end{center}
    \legend{Fonte: Autor}
\end{figure}
\begin{figure}[htb]
    \caption{\label{fig:tela-cadastros-armas}Autoform - Cadastros}
    \begin{center}
        \includegraphics[scale=0.5]{imagens/autoform-cadastros.png}
    \end{center}
    \legend{Fonte: Autor}
\end{figure}
\begin{figure}[htb]
    \caption{\label{fig:tela-configuracoes}Autoform - Configurações}
    \begin{center}
        \includegraphics[scale=0.5]{imagens/autoform-configuracoes.png}
    \end{center}
    \legend{Fonte: Autor}
\end{figure}



% Este capítulo possui também exemplos de como inserir código no texto.
% % ---
% \section{Vestibulum ante ipsum primis in faucibus orci luctus et ultrices
% posuere cubilia Curae}
% % ---

% \lipsum[21-22]

% \section{Inserindo código no texto}

% Abaixo são apresentados exemplos de códigos inseridos no texto usando o pacote listings.

% %Inserindo código diretamente no texto
% \begin{lstlisting}[language=c, caption={Exemplo de código C}, upquote=true]
% #include <stdio.h>

% int main(int argc, const char * argv[]) {
% struct pessoa{
% char nome[20];
% char sobreNome[20];
% unsigned short idade;
% char cpf[15];
% }p1;
% //struct pessoa p1;

% printf("Digite o nome: ");
% scanf("%s",p1.nome);
% printf("Digite o sobrenome: ");
% scanf("%s",p1.sobreNome);
% printf("Digite a idade: ");
% scanf("%hu",&p1.idade);
% printf("Digite o CPF: ");
% scanf("%s",p1.cpf);

% printf("\nO nome da pessoa eh: %s", p1.nome);
% printf("\nO sobrenome da pessoa eh: %s", p1.sobreNome);
% printf("\nA idade da pessoa eh: %d", p1.idade);
% printf("\nO CPF da pessoa eh: %s\n", p1.cpf);
% }
% \end{lstlisting}

% \lipsum[1]

% %Ao invés de inserir código diretamente no texto, é recomendável importar um arquivo com o código

% %Importanto arquivo de código
% \lstinputlisting[language=Java,caption={Exemplo de código java}]{codigos/Minhaclasse.java}

% O método abaixo mostra a string mensagem recebida como parametro e retorna um inteiro digitado pelo usuario. Através de tratamento de exceções, o método executará até que o usuário digite um inteiro válido.

% %Para inserir apenas algumas linhas do arquivo
% \lstinputlisting[language=Java, caption={Método LerInteiro}, firstline=39, lastline=54]{codigos/Ferramentas.java}

% \lipsum[2-3]
% ---


% ---
% Conclusão
% ---
\chapter{Conclusão}
O objetivo principal deste trabalho foi desenvolver uma solução que auxiliasse de forma eficaz os operadores da Brigada Militar ao realizar o processo de registro de produtos controlados no SIGMA através do AEL.

Para isso, foram levantadas as informações necessárias, através da elicitação de requisitos, pesquisa exploratória e análise documental para orientar a pesquisa em busca de uma solução eficiente que fosse segura, tivesse disponibilidade de acesso fácil, não armazenasse dados sensíveis, e ainda fosse capaz de minimizar o esforço empregado pelo operador para gerar o arquivo eletrônico.

Portanto o AutoForm atendeu aos objetivos específicos estabelecidos, proporcionando um preenchimento facilitado através de uma \textit{interface} intuitiva e amigável, disponível de fácil acesso através da \textit{web} e de forma segura, permitindo que diversos usuários realizem acesso simultâneo e ainda não armazenando dados sensíveis, pois o arquivo é gerado no computador do usuário.

A aplicação possibilita o cadastro de armas para todos os seus utilizadores, bem como o gerenciamento de usuários para o administrador viabilizando a supervisão por parte do administrador sobre quais pessoas têm acesso à aplicação.

\subsection{Resultados finais}

Vislumbrando-se uma estimativa de redução no tempo de preenchimento do formulário ao eleger uma arma previamente cadastrada para autopreencher 15 campos, seguida pela inserção de dados em apenas 24 campos remanescentes, contemplando inclusivamente os opcionais. 
Alternativamente, pode-se empregar o preenchimento de apenas 17 campos, sendo somente os obrigatórios,respectivamente, para a composição do arquivo eletrônico, outrora elaborado manualmente. Essa transição se baseia na totalidade de campos, quantificando em 39.


Logo esperasse que através deste trabalho tenha sido possível alavancar ainda mais a excelência operacional da Brigada Militar, contribuindo para a regulamentação dos produtos controlados e, consequentemente, para a segurança pública do estado do Rio Grande do Sul.

\section{Trabalhos futuros}
Nesta seção são descritos alguns trabalhos futuros que podem ser realizados para melhorar a aplicação.

Entre estas atualizações fica sugestão da implementação de alguns campos fixos no AEL como: tipo proprietário, orgão, orgão que publicou e profissão do proprietário, que não foram implementados por não serem obrigatórios, mas que podem ser definidos futuramente por se tratar de informações fixas inerentes a instituição, logo irá minimizar ainda mais a quantidade de campos a serem preenchidos.

Na página de configurações seria interessante adicionar uma opção para o administrador poder selecionar algum outro usuário comum e inserir essa permissão especial, para que assim possa ser delegado a outra pessoa a responsabilidade de gerenciar os demais usuários. Também seria interessante a possibilidade de um botão com a opção para cadastrar novos usuários através desta página.

Na página do AEL,fica a sugestão para adicionar um modal que permita visualizar e editar os registros temporários já salvos, para que assim o usuário possa alterar algum dado que tenha sido preenchido incorretamente, ou até mesmo excluir o registro temporário.


% ---


% ---


% ----------------------------------------------------------
% ELEMENTOS PÓS-TEXTUAIS
% ----------------------------------------------------------
\postextual
% ----------------------------------------------------------

% ----------------------------------------------------------
% Referências bibliográficas (Obrigatório)
% ----------------------------------------------------------
\bibliography{elementos-pos-textuais/referencias}


% ----------------------------------------------------------
% Glossário (Opcional)
% ----------------------------------------------------------
%
% Consulte o manual da classe abntex2 para orientações sobre o glossário.
%
%\glossary


% ----------------------------------------------------------
% Apêndices (Opcional)
% ----------------------------------------------------------
% ---
% Inicia os apêndices
% ---
% \begin{apendicesenv}

% \include{elementos-pos-textuais/apendices/apendice-a}
% \include{elementos-pos-textuais/apendices/apendice-b}

% \end{apendicesenv}
% ---


% ----------------------------------------------------------
% Anexos (Opcional)
% ----------------------------------------------------------
% ---
% Inicia os anexos
% ---
\begin{anexosenv}

% ---
\chapter{Portaria 136 - Anexo D - Página 26}
% ---
\section{AnexoD-A}
\includegraphics[scale=0.8]{imagens/AnexoA1-AnexoD-portaria-136}
\label{sec:anexoA1}




% ---
\chapter{Portaria 136 - Anexo D - Página 27}
% ---
\includegraphics[scale=0.7]{imagens/AnexoA2-AnexoD-portaria-136}


% ---
\chapter{\textit{Portaria 136 - Anexo D - Página 28}}
% ---
\includegraphics[scale=0.8]{imagens/AnexoA3-AnexoD-portaria-136}


% ---
\chapter{\textit{Portaria 136 - Anexo D.4}}\label{sec:anexoA4}
% ---
\includegraphics[scale=0.8]{imagens/AnexoA4-AnexoD-portaria-136}


% ---
\chapter{\textit{Referência Elogiosa BMRS - Parte nº 032/CMB/2023}}

\label{sec:anexoe}
% ---
\includegraphics[scale=0.8]{imagens/PORTARIA-INTERNA-BMRS.pdf}


% ---
\chapter{\textit{Referência Elogiosa BMRS - Parte nº 032/CMB/2023}}

\label{sec:anexof}
% ---
\includegraphics[scale=0.8]{imagens/PORTARIA-INTERNA-BMRS-2.pdf}


\end{anexosenv}


%---------------------------------------------------------------------
% INDICE REMISSIVO (Opcional)
%---------------------------------------------------------------------
%\phantompart
%\printindex
%---------------------------------------------------------------------

\end{document}
